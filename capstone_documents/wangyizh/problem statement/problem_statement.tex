\documentclass[10pt,onecolumn,journal,draftclsnofoot]{IEEEtran}
\usepackage[margin=0.75in]{geometry}
\renewcommand{\familydefault}{\rmdefault}

\usepackage{listings}
\usepackage{color}
\usepackage{hyperref}

\definecolor{dkgreen}{rgb}{0,0.6,0}
\definecolor{gray}{rgb}{0.5,0.5,0.5}
\definecolor{mauve}{rgb}{0.58,0,0.82}

\lstset{frame=tb,
  language=Bash,
  aboveskip=3mm,
  belowskip=3mm,
  showstringspaces=false,
  columns=flexible,
  basicstyle={\small\ttfamily},
  numbers=none,
  numberstyle=\tiny\color{gray},
  keywordstyle=\color{blue},
  commentstyle=\color{dkgreen},
  stringstyle=\color{mauve},
  breaklines=true,
  breakatwhitespace=true,
  tabsize=3
}

\begin{document}

\begin{titlepage}
\title{Develop an Internet of Things Irrigation Valve for a Real Company}
\author
{\IEEEauthorblockN{Yizheng Wang\\}
\IEEEauthorblockA{
Oregon State University\\
Computer Science 461\\
Fall 2017\\
}}
	\maketitle
	\vspace{4cm}
	\begin{abstract}
		\noindent Our project aims at set up a platform for a currently existing water control valve that can receive data from soil moisture sensors and control the valve with the platform. Our challenge is the data transmit between sensors and web platform. By doing this, farmers can have access to real-time soil moisture content of their fields, so they can decide where need water and where don’t need, it will make the control of valve more efficiency and save water from being wasting.
    The challenge can be overcome by adding a hub. The hub will collect data from sensors, then transmit the data to the web platform. It can also receive data from web platform, then send relay control signal to control the on/off of the valve. We will also need water-proof closure to protect the soil moisture sensors.

	\end{abstract}

\end{titlepage}

\newpage

\section{Definition and Description}
\par
For this project, we want to modify a currently-existing water control valve to measure soil moisture and connect to the web for automation and data sharing.
\par
The soil moisture will be measured by wireless soil moisture sensors, the data they get will be transmitted to a hub. The hub will send the data to an online app with data and the unique IDs of sensors. Then the app will display those data to the users. 
\par
The app will specify parameters like water event scheduling, soil moisture thresholds, etc. After users receive and view those data, they can decide what valve need to do base on the data. 
\par
And the decision will be sent back to the hub, which will send relay control signals to soil moisture sensors. The water control valve will change it behavior base on the decision information they receive.

\section{Proposed Solution}
\par
The soil moisture will be measured by wireless soil moisture sensors. The wireless soil moisture sensors will connect with Wi-Fi boards. Each soil moisture has a unique ID to distinguish them. They will be covered in water-proof closure. To make this water proof closure. We need to print it by 3D printing with software like CAD and Fusion 360. For every 15 minutes, soil moisture sensors will transmit the data they collected to the hub with their IDs with Wi-Fi boards. 
\\
\par
The hub will be used to receive data from soil moisture sensors, then upload those data to online, which will also include the IDs of the sensors. Hub may send data through ethernet shield. It may upload the data to the dashboard of Adafruit IO by using MQTT Api. Then those data will be synchronized with the web platform, which is the app we will program by IFTTT.
\par
The app will transfer and display it on one page. This cover many different information. Users can read it, and enter some parameters that can change the behavior of the water control valve. 
\par
The change of action that will make on valve will be send back through ethernet shield to the hub. Then the hub will send relay control signal to soil moisture sensors with IDs and the turn on/off option. After receive the signal, the valve will change their actions.
\newpage

\section{Performance metrics}
\par
To make sure we complete the project. We have at least test the following things and make sure they work well. 
\par
While the soil moisture sensors are covered by water-proof closures. We need to make sure the water-proof closures work, they should be able to cover the sensors and do not impact the functionality of the sensors. 
\par
According to the plan of the project, each soil moisture sensor has its unique ID, so the sensors should be able to send and receive the data that include their IDs correctly. Which means, they should be able to distinguish the IDs they have. Soil moisture are required to send data for every 15 minutes, so we may need to take some time to check if they can send data on time.
\par
The web client is used to control valve and display the data that it received from hub. So we may collect the data from soil moisture sensor by ourselves, then check if the data it displays on the screen is same as the data we collect. We can turn on/off each valve separately to make sure it can control the valve accurately.
\par
The hub is used to collect data from sensors with their IDs and transmit them to web platform. So if the sensors can receive control relay signal from web platform and take correct action, and the web platform can receive data from sensors, then that means the hub works properly.
\\
\par
If all of those parts work, that means we have completed the project.


\end{document}
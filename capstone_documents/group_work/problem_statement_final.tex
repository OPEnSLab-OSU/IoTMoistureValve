\documentclass[onecolumn, draftclsnofoot,10pt, compsoc]{IEEEtran}
\usepackage{graphicx}
\usepackage{url}
\usepackage{setspace}

%Personal imports
%\usepackage{cite}

\usepackage{geometry}
\geometry{textheight=9.5in, textwidth=7in}

% 1. Fill in these details
\def \CapstoneTeamName{		Group}
\def \CapstoneTeamNumber{		35}
\def \GroupMemberOne{			Christopher Carlsen}
\def \GroupMemberTwo{			Yizheng Wang}
\def \GroupMemberThree{			Peter Dorich}
\def \CapstoneProjectName{		Develop an Internet of Things Irrigation Valve}
\def \CapstoneSponsorCompany{	 	OSU \textbar Openly Published Environmental Sensing Lab (OPEnS) }
\def \CapstoneSponsorPerson{		Chet Udell}

% 2. Uncomment the appropriate line below so that the document type works
\def \DocType{		Problem Statement
	%Requirements Document
	%Technology Review
	%Design Document
	%Progress Report
}

\newcommand{\NameSigPair}[1]{\par
	\makebox[2.75in][r]{#1} \hfil 	\makebox[3.25in]{\makebox[2.25in]{\hrulefill} \hfill		\makebox[.75in]{\hrulefill}}
	\par\vspace{-12pt} \textit{\tiny\noindent
		\makebox[2.75in]{} \hfil		\makebox[3.25in]{\makebox[2.25in][r]{Signature} \hfill	\makebox[.75in][r]{Date}}}}
% 3. If the document is not to be signed, uncomment the RENEWcommand below
%\renewcommand{\NameSigPair}[1]{#1}

%%%%%%%%%%%%%%%%%%%%%%%%%%%%%%%%%%%%%%%
\begin{document}
	\begin{titlepage}
		\pagenumbering{gobble}
		\begin{singlespace}
			\includegraphics[height=4cm]{coe_v_spot1}
			\hfill 
			% 4. If you have a logo, use this includegraphics command to put it on the coversheet.
			%\includegraphics[height=4cm]{CompanyLogo}   
			\par\vspace{.2in}
			\centering
			\scshape{
				\huge CS461 Capstone \DocType \par
				{\large\today - Fall Term}\par
				\vspace{.5in}
				\textbf{\Huge\CapstoneProjectName}\par
				\vfill
				{\large Prepared for}\par
				\Huge \CapstoneSponsorCompany\par
				\vspace{5pt}
				{\Large\NameSigPair{\CapstoneSponsorPerson}\par}
				{\large Prepared by }\par
				Group\CapstoneTeamNumber\par
				% 5. comment out the line below this one if you do not wish to name your team
				%\CapstoneTeamName\par 
				\vspace{5pt}
				{\Large
					\NameSigPair{\GroupMemberOne}\par
					\NameSigPair{\GroupMemberTwo}\par
					\NameSigPair{\GroupMemberThree}\par
				}
				\vspace{20pt}
			}
			\begin{abstract}
				% 6. Fill in your abstract   --- TODO 
				This document covers the the global issue of water shortage as it pertains to the field of agriculture, and a proposed solution to it.
				Agriculture makes up the vast majority of water usage in most parts of the world, and this projects intention is to develop a solution to that issue.
				The contents within will outline the overall problem, and the agreed upon terms of completion between our group and its client.
			\end{abstract}     
		\end{singlespace}
	\end{titlepage}
	\newpage
	\pagenumbering{arabic}
	\tableofcontents
	% 7. uncomment this (if applicable). Consider adding a page break.
	%\listoffigures
	%\listoftables
	\clearpage
	
	% 8. now you write!
	\section{Problem}
	
	%Agriculture consumes 70% of the planet’s accessible freshwater. Climate change continues to expand water-strained regions around the world. When farms over water crops, bad things happen.
	\subsection{Overview}
	Agricultural over-use of water resources is an ever increasing threat to overall global well being.
	When adding climate change issues to this context, these concerns grow exponentially.
	Areas who's water resources are already strained will find themselves in an increasingly difficult situation as agriculture needs continue to increase, and climate changes persist.
	Directly speaking, agricultural irrigation can account for up to 70\% of water resource consumption in a given area.%\cite{global_agriculture}
	This is the problem that this project aims to address.
	
	More specifically, the problem that our client and this group aim to address is improving upon an existing model of an irrigation valve originally designed by P\&R Surge Systems, Inc.
	The current valve system functions through the use of a solar-powered, timer-based device that operates the valve.
	However this system still leads to issues where crops are being unnecessarily watered in areas where the soil has retained sufficient moisture and needs to additional water at that time.
	Additionally, this solution needs to be implemented in as cost-efficient a way as possible, to maximize its accessibility and usage.
	
	\section{Solution proposal}
	%P&R Surge Systems wants to partner with YOU to design the next smart precision-irrigation system for the agricultural industry. Your team will work directly with Dr. Chet Udell of OSU's Openly Published Environmental Sensing Lab and David Jurado, Operations Manager of P&R, to create an Open Source, modular, plug-and-play, internet-connected water valve system with soil moisture sensors. With hundreds of these in a field, you'll ensure water is only used where it is needed and that farmers and machines alike have access to real-time soil moisture content of their fields.
	
	\subsection{Overview}
	In seeking to solve the aforementioned water usage problem, our client has envisioned a solution that would allow agriculturalists access to soil moisture data, through use of an Internet of Things style device.
	This device would record soil moisture data, and then transmit the data through a wireless medium back to a home-station, which would then process and record the data, while uploading it to the internet to share.
	This data would be utilized by the software, running it through decision-making algorithms which would allow automated, wireless control of the valves, getting water only to the areas that need it.
	These algorithms could be either built-in, or user-defined, using the moisture data to make more informed and sustainable choices regarding the timing of the irrigation of their crops.
	
	An efficient and economically-viable water transport and valve system has already been acquired by our client for use in this project.
	The valve-plate that directs water flow within this pipe system is controlled by a simple spur-style gear head construct.
	This gear is then connected to the solar-powered control device, allowing mechanical control of the valve-plate from the pipes exterior.
	The client would like to create an external hardware add-on that physically mounts and communicates instructions to the valve-control device.
	This hardware add-on would also be the communication device that relays the data from separate soil moisture sensors back to a home-station.
	The user would be able to track data about soil moisture in their various agricultural plots, through use of either a web or application interface.
	
	Our client's desired solution to this issue consists of three main parts, that will all operate in unison to achieve the previously mentioned goals.
	
	\subsection{Part one - Sensor and communication devices}
	Required skills:
	\begin{itemize}
		\item C programming proficiency
		\item Understanding of digital logic and microprocessors
		\item Experience with circuits and sensors
		\item Wireless communications understanding
	\end{itemize}
	The first production of this project will be the creation of multiple portable devices.
	The devices will use microprocessors, specifically Arduino boards.
	These Arduino will be broken up into two categories, the soil moisture sensor, and the wireless communication devices.
	The moisture sensing board will likely have a physical connection to the communication device that is within the external enclosure.
	The Arduino within the external enclosure will be used to relay the soil data to the base station, and additionally will receive the commands to actually control the valve-plate.
	The optimal design for these devices will need to take cost heavily into consideration, to maintain the greatest level of accessibility.
	These devices will be programmed using the C language.
	
	\subsection{Part two - Wireless Hub}
	Required skills:
	\begin{itemize}
		\item Proficiency with Arduino, and Fusion 360
		\item Understanding of materials and data transfer
	\end{itemize}
	
	The next step is to develop a wireless hub to transmit soil data to a web application. 
    The hub will be in control of all conditional signals to the valve as well. When commands are sent to the wireless hub, they will be verified with a unique I.D. 
	In addition, the soil moisture monitors and the mounted information relay will need to be housed within waterproof enclosures, as they will always be directly exposed to the elements.
	These enclosures will be designed in a CAD environment (specifically Fusion 360, as per the client) and will then be 3D printed.
	The materials for these enclosures should be durable enough to withstand exposure, while also trying to keep materials costs to a minimum.
	
	\subsection{Part three - Web and/or application interface}
	Required skills:
	\begin{itemize}
		\item Web-development proficiency (HTML/CSS/JavaScript/Node/etc.)
		\item IFTTT (or comparable) proficiency
		\item Web applet development experience
		\item Wireless communications understanding
		\item Mobile development skills
	\end{itemize}
	The final piece of this project will involve the development of a user-readable software platform.
	In particular, this software be used to track local soil moisture levels, and allow agriculturalists to choose the most optimal times to run the irrigation systems that water their crops.
	This software should be deployed to both a web-based platform, as well as a desktop application platform.
	Optimally, the client would also like a mobile app developed for Android, iOS, or potentially both.
	The data that this software handles should also, preferentially, be available online for others to view.
	
	\pagebreak
	\section{Performance metrics}
	This project will be considered successfully complete upon delivery of the following:
	\begin{itemize}
		\item Proof-of-concept Arduino devices:
		\begin{itemize}
			\item An Arduino board that can functionally read data from a soil moisture sensor, as well as transmit the data back. Also must be capable of receiving instructions to turn valve.
			\item Wireless communication hub for relaying data and instructions. Must be able to receive data from sensor devices, as well as relay it to the internet. Also makes decisions on which valves to activate and sends instructions to sensor devices accordingly.
		\end{itemize}
		\item Hardware enclosures that:
		\begin{itemize}
			\item Are durable and water/elements proof.
			\item Have materials and design chosen with affordability in mind.
		\end{itemize}
		\item Proof-of-concept software suite with:
		\begin{itemize}
			\item Soil-moisture data tracking and presentation abilities
			\item Remote control capabilities of irrigation valves
			\item Internet data-sharing of collected data
			\item Intelligently designed algorithms that can be used to automate valve control        
		\end{itemize}
		\item All items must be functional and field demonstration ready
	\end{itemize}
	\newpage
	%\bibliographystyle{IEEEtran}
	%\bibliography{references}
	
\end{document}
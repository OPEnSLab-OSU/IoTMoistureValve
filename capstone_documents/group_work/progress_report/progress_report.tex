\documentclass[onecolumn, draftclsnofoot,10pt, compsoc]{IEEEtran}
\usepackage{graphicx}
\usepackage{url}
\usepackage{setspace}

%Personal imports
%\usepackage{cite}
\newcommand{\subparagraph}{}
\usepackage{titlesec}
\usepackage{hyperref}
\usepackage{xcolor}

%Change link colors
\hypersetup{
    colorlinks=true,
    linkcolor=black,
    citecolor=black,
    filecolor=black,
    urlcolor=black,
}

\usepackage{geometry}
\geometry{textheight=9.5in, textwidth=7in}

\titleclass{\subsubsubsection}{straight}[\subsection]
\titleclass{\subsubsubsubsection}{straight}[\subsection]

\newcounter{subsubsubsection}[subsubsection]
\newcounter{subsubsubsubsection}[subsubsubsection]
\renewcommand\thesubsubsubsection{\thesubsubsection.\arabic{subsubsubsection}}
\renewcommand\thesubsubsubsubsection{\thesubsubsubsection.\arabic{subsubsubsubsection}}
\renewcommand\theparagraph{\thesubsubsubsection.\arabic{paragraph}} % optional; useful if paragraphs are to be numbered

\titleformat{\subsubsubsection}
  {\normalfont\normalsize\bfseries}{\thesubsubsubsection}{1em}{}
\titlespacing*{\subsubsubsection}
{0pt}{3.25ex plus 1ex minus .2ex}{1.5ex plus .2ex}
\titleformat{\subsubsubsubsection}
  {\normalfont\normalsize\bfseries}{\thesubsubsubsubsection}{1em}{}
\titlespacing*{\subsubsubsubsection}
{0pt}{3.25ex plus 1ex minus .2ex}{1.5ex plus .2ex}


\makeatletter
\renewcommand\paragraph{\@startsection{paragraph}{5}{\z@}%
  {3.25ex \@plus1ex \@minus.2ex}%
  {-1em}%
  {\normalfont\normalsize\bfseries}}
\renewcommand\subparagraph{\@startsection{subparagraph}{6}{\parindent}%
  {3.25ex \@plus1ex \@minus .2ex}%
  {-1em}%
  {\normalfont\normalsize\bfseries}}
\def\toclevel@subsubsubsection{4}
\def\toclevel@subsubsubsubsection{5}
\def\toclevel@paragraph{6}
\def\toclevel@paragraph{7}
\def\l@subsubsubsection{\@dottedtocline{4}{11em}{5em}}
\def\l@subsubsubsubsection{\@dottedtocline{5}{16em}{6em}}
\def\l@paragraph{\@dottedtocline{5}{10em}{5em}}
\def\l@subparagraph{\@dottedtocline{6}{14em}{6em}}
\makeatother

\setcounter{secnumdepth}{5}
\setcounter{tocdepth}{5}


% 1. Fill in these details
\def \CapstoneTeamName{     Group}
\def \CapstoneTeamNumber{       35}
\def \GroupMemberOne{           Christopher Carlsen}
\def \GroupMemberTwo{           Yizheng Wang}
\def \GroupMemberThree{         Peter Dorich}
\def \CapstoneProjectName{      Developing an Internet of Things Irrigation Valve}
\def \CapstoneSponsorCompany{       OSU \textbar\hspace{.05in} Openly Published Environmental Sensing (OPEnS) Lab}
\def \CapstoneSponsorPerson{        Chet Udell}

% 2. Uncomment the appropriate line below so that the document type works
\def \DocType{      %Problem Statement
    %Requirements Document
    %Technology Review
    %Design Document
    Progress Report
}

\newcommand{\NameSigPair}[1]{\par
    \makebox[2.75in][r]{#1} \hfil   \makebox[3.25in]{\makebox[2.25in]{\hrulefill} \hfill        \makebox[.75in]{\hrulefill}}
    \par\vspace{-12pt} \textit{\tiny\noindent
        \makebox[2.75in]{} \hfil        \makebox[3.25in]{\makebox[2.25in][r]{Signature} \hfill  \makebox[.75in][r]{Date}}}}
% 3. If the document is not to be signed, uncomment the RENEWcommand below
\renewcommand{\NameSigPair}[1]{#1}

%%%%%%%%%%%%%%%%%%%%%%%%%%%%%%%%%%%%%%%
\begin{document}
    \begin{titlepage}
        \pagenumbering{gobble}
        \begin{singlespace}
            \includegraphics[height=4cm]{coe_v_spot1}
            \hfill 
            % 4. If you have a logo, use this include graphics command to put it on the coversheet.
            %\includegraphics[height=4cm]{CompanyLogo}   
            \par\vspace{.2in}
            \centering
            \scshape{
                \huge CS461 Capstone \DocType \par
                {\large\today - Fall Term}\par
                \vspace{.5in}
                \textbf{\Huge\CapstoneProjectName}\par
                \vfill
                %{\large Prepared for}\par
                %\Huge \CapstoneSponsorCompany\par
                %\vspace{10pt}
                %{\Large\NameSigPair{\CapstoneSponsorPerson}\par}
                {\large Prepared and issued by }\par
                Group\CapstoneTeamNumber\par
                % 5. comment out the line below this one if you do not wish to name your team
                %\CapstoneTeamName\par 
                \vspace{5pt}
                {\Large
                    \NameSigPair{\GroupMemberOne}\par
                    \NameSigPair{\GroupMemberTwo}\par
                    \NameSigPair{\GroupMemberThree}\par
                }
                \vspace{20pt}
            }
            \begin{abstract}
                % 6. Fill in your abstract   --- TODO 
                This document is a .
            \end{abstract}     
        \end{singlespace}
    \end{titlepage}
    \newpage
    \pagenumbering{arabic}
    \tableofcontents
    % 7. uncomment this (if applicable). Consider adding a page break.
    %\listoffigures
    %\listoftables
    \clearpage
    
    % 8. now you write!
    \section{Recap of Purpose and Goals}
    The Internet of Things Irrigation Valve project is an open-source, proof of concept project that aims to create an internet-connected, irrigation system control suite.
    The device is being developed with farmers/agriculturists in mind, and is an attempt to design a more sustainable and efficient method of watering specific, targeted plots of land.
    The current method of irrigation employed by the majority of agriculturists generally relies on either the user physically engaging/disengaging the irrigation valves,
    
    Both of these solutions require that the users travel to each individual valve if changes are desired.
    While more complex and advanced systems do exist, at the time of this writing, they are still rather uncommon and are generally expensive.
    This leads to the primary goals of this project, which are to:
    \begin{enumerate}
    \item create an Internet of Things connected irrigation system. 
    \item expand on the current standard solutions of irrigation, allowing for valve control via time, soil moisture content or both.
    \item centralize the valve control system to a single, wireless remote location so that users can easily see the state of their various fields, and make informed decisions about best irrigation schemes, without needing to travel to each physical location.
    \item keep the developed software package open-source, allowing users access to cheaper options than would generally be found.
    \end{enumerate}

    \section{Project Progress and Current State}
    \subsection{Week 1}
    This was a fairly simple week, being the first week of the course.
    We worked on creating personal resumes for the class, but the bulk of the week was spent looking through the Capstone project options and trying to decide which projects to apply for. 
    
    \subsection{Week 2}
    Having just been assigned our projects, this week was mostly spent trying to meet up and get organized.
    We sent out our first e-mail to the client on Wednesday, and then the group met in person after class on Thursday.
    After initial e-mail contact with our client, Chet Udell, we scheduled to meet on Friday afternoon for a simple introduction to the project, and a meet and greet.
    Some discussion happened regarding roles in the project, and the general format of the Problem Statement was laid out.
    \subsection{Week 3}
    
    
    \subsection{Week 4}
    
    \subsection{Week 5}
    
    \subsection{Week 6}
    
    \subsection{Week 7}
    
    \subsection{Week 8}
    
    \subsection{Week 9}
    
    \subsection{Week 10}
    
    \section{Problems and Solutions}
    
    \section{Interesting Tidbits}
    
    \pagebreak
    \section{10 Week Retrospective}
    \vspace{.1in}
    \begin{center}
    \begin{tabular}{| p{0.3\linewidth}| p{0.3\linewidth}| p{0.3\linewidth}|}
    Positives & Deltas & Actions \\
    \hline
    {Had an interesting Skype call with secondary client, P\&R Surge Systems, and were able to talk with them regarding their interest in and thoughts about the project.} & - & - \\
    \hline
    - & {Some lack of clarity regarding } & - \\
    %Copy the following two lines and add above to add new row.
    %\hline
    %- & - & - \\
    \hline
    \end{tabular}
    \end{center}

    %-----------------
    %\pagebreak
    %\nocite{*} %TODO - Remove this if citing things.
    %\bibliographystyle{IEEEtran}
    %\bibliography{references}    
    
\end{document}

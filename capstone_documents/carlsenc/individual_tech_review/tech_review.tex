\documentclass[onecolumn, draftclsnofoot,10pt, compsoc]{IEEEtran}
\usepackage{graphicx}
\usepackage{url}
\usepackage{setspace}

%Personal imports
%\usepackage{cite}
\newcommand{\subparagraph}{}
\usepackage{titlesec}
\usepackage{hyperref}
\usepackage{xcolor}

%Change link colors
\hypersetup{
    colorlinks=true,
    linkcolor=black,
    citecolor=black,
    filecolor=black,
    urlcolor=black,
}

\usepackage{geometry}
\geometry{textheight=9.5in, textwidth=7in}

\titleclass{\subsubsubsection}{straight}[\subsection]
\titleclass{\subsubsubsubsection}{straight}[\subsection]

\newcounter{subsubsubsection}[subsubsection]
\newcounter{subsubsubsubsection}[subsubsubsection]
\renewcommand\thesubsubsubsection{\thesubsubsection.\arabic{subsubsubsection}}
\renewcommand\thesubsubsubsubsection{\thesubsubsubsection.\arabic{subsubsubsubsection}}
\renewcommand\theparagraph{\thesubsubsubsection.\arabic{paragraph}} % optional; useful if paragraphs are to be numbered

\titleformat{\subsubsubsection}
  {\normalfont\normalsize\bfseries}{\thesubsubsubsection}{1em}{}
\titlespacing*{\subsubsubsection}
{0pt}{3.25ex plus 1ex minus .2ex}{1.5ex plus .2ex}
\titleformat{\subsubsubsubsection}
  {\normalfont\normalsize\bfseries}{\thesubsubsubsubsection}{1em}{}
\titlespacing*{\subsubsubsubsection}
{0pt}{3.25ex plus 1ex minus .2ex}{1.5ex plus .2ex}


\makeatletter
\renewcommand\paragraph{\@startsection{paragraph}{5}{\z@}%
  {3.25ex \@plus1ex \@minus.2ex}%
  {-1em}%
  {\normalfont\normalsize\bfseries}}
\renewcommand\subparagraph{\@startsection{subparagraph}{6}{\parindent}%
  {3.25ex \@plus1ex \@minus .2ex}%
  {-1em}%
  {\normalfont\normalsize\bfseries}}
\def\toclevel@subsubsubsection{4}
\def\toclevel@subsubsubsubsection{5}
\def\toclevel@paragraph{6}
\def\toclevel@paragraph{7}
\def\l@subsubsubsection{\@dottedtocline{4}{11em}{5em}}
\def\l@subsubsubsubsection{\@dottedtocline{5}{16em}{6em}}
\def\l@paragraph{\@dottedtocline{5}{10em}{5em}}
\def\l@subparagraph{\@dottedtocline{6}{14em}{6em}}
\makeatother

\setcounter{secnumdepth}{5}
\setcounter{tocdepth}{5}


% 1. Fill in these details
\def \CapstoneTeamName{     Group}
\def \CapstoneTeamNumber{       35}
\def \GroupMemberOne{           Christopher Carlsen - Soil Moisture Sensor and Valve Control}
\def \GroupMemberTwo{           Yizheng Wang}
\def \GroupMemberThree{         Peter Dorich}
\def \CapstoneProjectName{      Develop an Internet of Things Irrigation Valve}
\def \CapstoneSponsorCompany{       OSU \textbar\hspace{.05in} Openly Published Environmental Sensing (OPEnS) Lab}
\def \CapstoneSponsorPerson{        Chet Udell}

% 2. Uncomment the appropriate line below so that the document type works
\def \DocType{      %Problem Statement
    %Requirements Document
    Technology Review
    %Design Document
    %Progress Report
}

\newcommand{\NameSigPair}[1]{\par
    \makebox[2.75in][r]{#1} \hfil   \makebox[3.25in]{\makebox[2.25in]{\hrulefill} \hfill        \makebox[.75in]{\hrulefill}}
    \par\vspace{-12pt} \textit{\tiny\noindent
        \makebox[2.75in]{} \hfil        \makebox[3.25in]{\makebox[2.25in][r]{Signature} \hfill  \makebox[.75in][r]{Date}}}}
% 3. If the document is not to be signed, uncomment the RENEWcommand below
\renewcommand{\NameSigPair}[1]{#1}

%%%%%%%%%%%%%%%%%%%%%%%%%%%%%%%%%%%%%%%
\begin{document}
    \begin{titlepage}
        \pagenumbering{gobble}
        \begin{singlespace}
            \includegraphics[height=4cm]{coe_v_spot1}
            \hfill 
            % 4. If you have a logo, use this include graphics command to put it on the coversheet.
            %\includegraphics[height=4cm]{CompanyLogo}   
            \par\vspace{.2in}
            \centering
            \scshape{
                \huge CS461 Capstone \DocType \par
                {\large\today - Fall Term}\par
                \vspace{.5in}
                \textbf{\Huge\CapstoneProjectName}\par
                \vfill
                %{\large Prepared for}\par
                %\Huge \CapstoneSponsorCompany\par
                %\vspace{10pt}
                %{\Large\NameSigPair{\CapstoneSponsorPerson}\par}
                {\large Prepared by }\par
                Group\CapstoneTeamNumber\par
                % 5. comment out the line below this one if you do not wish to name your team
                %\CapstoneTeamName\par 
                \vspace{5pt}
                {\Large
                    \NameSigPair{\GroupMemberOne}\par
                    %\NameSigPair{\GroupMemberTwo}\par
                    %\NameSigPair{\GroupMemberThree}\par
                }
                \vspace{20pt}
            }
            \begin{abstract}
                % 6. Fill in your abstract   --- TODO 
                This document is a review of the technology that will be used through out the Irrigation Valve project.
                Contained within will be a more granular analysis of potential hardware for the project, as well as a compare/contrast style discussion about possible alternatives.
            \end{abstract}     
        \end{singlespace}
    \end{titlepage}
    \newpage
    \pagenumbering{arabic}
    \tableofcontents
    % 7. uncomment this (if applicable). Consider adding a page break.
    %\listoffigures
    %\listoftables
    \clearpage
    
    % 8. now you write!
    \section{Piece One - Data and Command Handling Device}
    \subsection{Overview}
    As the work horse of this segment of the project, this device will need to be capable of frequently recording data from the the selected soil moisture recording device (including packaging the recorded data for transmission to the hub device).
    It will also be responsible for the storage and handling of received instructions on how/when to operate the attached irrigation valve.
    \subsection{Criteria}
    There aren't a large amount of criteria that this particular device must meet, but the few criteria that it does have are fairly crucial to the success of the project.
    The first major criterion for this segment that needs to be met without fail is good power management/low power usage. This is more or less an absolute requirement of this device, as it must be able to remain in-field for extended periods of time without needing attention/maintenance from the user.\par
    A second criterion is that the chosen device for this piece has enough memory to store and handle any data that it retrieves from the sensor.
    %Get more criterion?
    
    \subsection{Potential Choices}
    \subsubsection{Choice One - Raspberry Pi Zero}
    The Raspberry Pi Zero is an interesting choice compared to the other two devices in this section, for a couple of reasons.
    It has a much faster processor (1GHz) and more on-board memory (512MBs) than either of the other two devices offer, which would definitely allow for extremely fast data handling.
    That being said, the power consumption for this board is quite a bit higher than the other options listed here.
    An average estimate given was listed at about 80mA with no peripherals active.\cite{rpi_power}
    The other major issue is that there really isn't a reasonable way to put the device to sleep and subsequently wake it up, like other devices have, as it wasn't an intended design.
    \subsubsection{Choice Two - Atmel SAML21 series}
    This board's strength lies in its ridiculously low power consumption.
    The SAML21 series touts that it utilizes only 35mA of power during active processing time\cite{digikey_saml21}, and anecdotally it was stated that the device is able to run on on 200nA of power in sleep mode.\cite{atmel_anecdote}
    It sports a 48MHz processor and 256KB of Flash memory as well as 40KB of SRAM, so program storage and runtime might be a little tight on memory but not still totally usable.\par
    One negative mark that the Atmel series has against it, however, is general peripheral and community support that many of the "bigger name" options usually come with.
    \subsubsection{Choice Three - Adafruit Feather}
    The Feather series of devices for the most part have the power efficiency bit going for them, but do lack a bit in the processing power and memory departments.
    Depending on the model selected, either the 32u4 or the M0 models, an 8Mhz or 48Mhz processor is used respectively.\par
    Something of note is that what the Feather lacks in areas of speed and memory, it substantially makes up for with a large support and community network that is very useful for trouble shooting issues.
    It also has a much larger base of peripherals than most other microcontroller options, due to its popularity.
    
    \subsection{Discussion}
    %"Use comparison terms like "in contrast" and "compared to"
    The choice for this particular piece of hardware was a difficult one to make.
    With such an immense pool microcontroller boards to choose from, many of them obscure brands, or having specifications that are well above/below what are required for this project, it was a bit of a chore to wade through them all.
    With that, comparing these boards came down mostly to the boards ability to run with serious power efficiency, while still having enough processing power and memory to meet processing needs.
    The Atmel SAML21 series of devious won out completely in comparison to the Raspberry Pi Zero, with little to no competition on the power utilization front.
    The Pi Zero was hurt especially bad due to its lack of a useful "sleep" state to put it into a low-power mode that it could recover from gracefully.
    While its processing power and memory are great, they are above and beyond what is truly required for this project and just don't make up for the power consumption.\par
    In contract, the Feather series put up a little bit more of a fight, keeping fairly low power usage rates both during run-time and during sleep.
    The Feather series did lack a bit in the processing power and memory department however.
    This issue could be shored up a bit if the M0 variety of Feather boards was used, as it has a 48MHz processor, however this necessarily meant an increase in power, which put a dent in its overall competitiveness.   
    
    \subsection{Conclusion}
    %"We chose X because..."; can include a simplified table.
    After looking through the various options, we concluded that the Atmel SAML21 series of micro-control boards would likely be a great candidate for this project. With its super-low power consumption during both sleep mode and active processing, as well as its 48MHz processor, it is a good fit for the project overall.
    With that being said, we have chosen to use the Adafruit Feather series of devices to meet this piece of the project's needs as its use has been specifically requested. The client has many of the devices on hand, and has found many useful code repositories that should allow for smoother interaction with necessary peripheral devices. These device libraries would have been otherwise difficult to find, or potentially would have had to have been programmed from scratch.
    
    \section{Piece Two - Soil Moisture Sensing Device}
    \subsection{Overview}
    The primary form of data collection for this project, this device will be the one that gathers all of the moisture data used to make decisions on this project.
    The sensor device will be left in the soil near the different valve points.
    Based on the data that this device collects, the attached valve can make informed decisions about whether or not it should open up, or remain closed.
    
    \subsection{Criteria}
    Fairly light criteria for this piece overall.
    The two main important factors for deciding this device are its accuracy and its reliability.\par
    It will be left in a field for the entirety of its use, and thus needs to be as resistant to the elements as possible.
    \subsection{Potential Choices}
    \subsubsection{Choice One - Decagon GS3 Series Soil Moisture Sensor}
    The Decagon series of soil moisture sensors are fairly impressive.
    In particular the most impressive part of the GS3 is its level of achieved accuracy. The GS3 also has a low power usage overall, using roughly only 25mA during its read.
    
    \subsubsection{Choice Two - Phantom YoYo Soil Moisture Sensor}
    The Phantom YoYo is an alternative choice to this project.
    The largest thing it has going for it is it's low power consumption on read, listed at being below 20mA of required power.
    It also is gold-immersed/plated which is a good preventative measure against oxidation, one of the biggest problems that moisture sensors suffer from.
    It is also startlingly cheap, coming in at only \$1.15 on one listing in particular.
    \subsubsection{Choice Three - SparkFun Soil Moisture Sensor}
    This SparkFun soil moisture sensors are a hobbyist line of moisture sensing hardware.
    This particular line is not meant for any especially intense usage, but has the same gold-immersion plating that the above mention Phantom YoYo does, to increase it's longevity.
    It also has an very cheap price, with the price listed from the SparFun website at \$4.95
    Probably the largest benefit that this version of sensor has going for it is its ease of compatibility with Arduino (and rebranded) devices.
    
    \subsection{Discussion}
    %"Use comparison terms like "in contrast" and "compared to"
    The considerations for this particular piece really is not much of a contest.
    While the Decagon GS3 device is definitely more expensive than the other two options considered(\textasciitilde \$200.00 compared to <\$5.00), the quality of data provided, along with the sturdy build and covered cabling that the GS3 offers make it superior in all important areas of consideration.
    
    \subsection{Conclusion}
    %"We chose X because..."; can include a simplified table.
    We chose the Decagon GS3 as our primary device in this category because of its overall dominance in the moisture sensing field. Although price is definitely a bit higher than would prefer, accuracy and reliability of data are key in this project and the GS3 offers that better than anything else on the market at the time of this writing.
    
    \section{Piece Three - Wireless Communication Standard/Device}
    \subsection{Overview}
    This segment will deal with the wireless communication standard what will be used between the valve control devices and the home hub device.
    The general premise of this piece is that it will be the lifeline that passes communications back and forth between the device and the endpoint user.
    (An important side-note that needs to be taken into consideration while reading through this section is that all of these standards can have their ranges increased through use of additional power/equipment, but in the interest of keeping this comparison as clear and balanced as possible, the bare-bones options will be considered here.)
    \subsection{Criteria}
    The wireless mode of communication for this portion of the device will take into consideration both the physical hardware itself, but more importantly the standard of communication itself.
    The first major issue that was assessed for this piece is transfer range.
    As these devices will be left in various fields across the user's agricultural land, they will need to achieve ranges fitting to those distances.\par
    The next consideration, that will be part of a larger overall concern for the device as a whole, is power consumption.
    As mentioned previously, devices will be left in various locations across farm fields, and as such it is fairly critical that they be as power efficient as possible while still being effective at communication.\par
    
    \subsection{Potential Choices}
    \subsubsection{Choice One - GSM/CDMA/IDEN}
    The first thought for this project, cell standards meet the range needs of this project by far.
    The various standards used by cellular bands offers superior range over the other options in this section, which would be largely important.
    However, these ranges have a particular limitation that could be an issue; although cellular bands can regularly reach distances between 20-30 miles, there is a huge level of variance in their range capabilities, some reaching as far as 40 miles, and others as little as 1-2 miles.
    This is highly dependent on the level of electronic and physical interference in the area, as well as the power supplied to the local cell tower.\cite{cell_wiki}\cite{hsw_cell}\par
    The next major consideration is that power consumption with a cellular band transceiver would be much larger overall.
    The average cellphone uses between 0.6 and 3 watts, depending largely on range to the receiver.\cite{hsw_cell}
    While this issue isn't exclusive to this wireless medium, because the ranges are significantly larger than the other options in this list, it will need to be taken into consideration when looking at power consumption.\par
    A smaller, but potentially significant consideration with cellular bands is that use of them would require some kind of subscription fee with a selected service provider, which adds to the cost overhead, as well as reliance on the selected service's uptime, rather than our own device's abilities.
    
    \subsubsection{Choice Two - RFM95 LoRa}
    A clear option for the wireless communication standard for this project, RFM95, being a "long range" radio variant, definitely meets this project's range needs without much issue.
    The RFM95 series of wireless communication is able reach ranges of 500m to 2km, with a decent balance of range and data throughput, while still maintaining rather low power consumption overall.
    In particular, and probably the largest boon of this particular option is its overall very low power consumption.
    In one test, a 20 byte payload was sent with only 130mA of power usage over a 70ms period using an Adafruit Feather LoRa RFM95.\cite{ladyada_pdf}    
    
    \subsubsection{Choice Three - WiFi / nRF24L01+}
    WiFi, nRF24L01+ and other 2.4GHz devices were also considerations for this project.
    These devices have the advantage have having the largest data throughput ability. 
    Depending on the standard, the estimate data-transfer rates can theoretically range from 11Mbps to 100+Mbps, but have actual recorded values from 2-50Mbps in most real world applications.\cite{speedguide}
    A major setback for these types of devices is required power.
    WiFi in general does not transmit nearly as far as the other mentioned options because of being at a higher point in the wave-length spectrum, as this option is generally in the 2.4GHz range or potentially at 5GHz depending on device choice.
    To achieve material penetration and distance levels similar to the previous options this would require more power than the others.
    \subsection{Discussion}
    %"Use comparison terms like "in contrast" and "compared to"
    Over all there are quite a few potential options for wireless communication.
    Some of the options not included were things like Bluetooth, and even satellite communication.
    These options could potentially achieve the same goals as the listed three, but generally fell well outside the applicable spectrum, or achieving the desired goal would have required significant alterations.\par
    That being said, the choices above are were selected because of their general relevance in coverage area, data throughput, and so on.
    With regards to transfer range capabilities, the cellular device standards win by a rather large margin.
    With transfer distances frequently being on the order of 10-20KM, and the shorter distances being around 2KMs with heavy interference/obstruction.
    WiFi/nRF24L01+ and RFM95-LoRa just don't really even compare with this kind of range, as getting them to send signals this far (if even possible) would require specialized hardware and a significant increase in power consumption.
    The RFM95-LoRa standard can, with a high-power, uni-directional antenna, reach distances of around 2KMs, but this is definitely pushing its limits.\par
    The other primary issue of this piece of the device being power consumption, the RFM95 standard definitely won out in contrast to both the WiFi and cellular fields.
    While all of these devices could potentially be put in a "low power" state, the average usage of power on send-time appears to be lowest with the RFM95 LoRa style devices.
    Cellular devices, as mentioned above, require between .6-3A for cellular transmission depending heavily on the device and also the distance to the nearest cell tower.
    The WiFi standard runs much closer to the RFM95-LoRa option, with WiFi generally using about 150mA to send and 50mA to receive across most micro-controller style devices.
    However, the RFM95 LoRa option, has certain device options which allow sending at 130mA roughly, and receiving at a pretty amazing 13mA.\par
    The area in which WiFi wins out when compared to the other two options is in sheer data throughput.
    However, for the needs of this project, we will be transmitting fairly small packets of data between devices, and WiFi's data transfer speeds are just well, well above anything that will be needed for this project.
    
    \subsection{Conclusion}
    %"We chose X because..."; can include a simplified table.
    In conclusion, we chose to use the RFM95/LoRa standard radio because it performs very well in regards to power consumption, while still being able to reach larger ranges than your average WiFi device.
    
    %-----------------
    \pagebreak
    \nocite{*} %TODO - Remove this if citing things.
    \bibliographystyle{IEEEtran}
    \bibliography{references}
    
\end{document}